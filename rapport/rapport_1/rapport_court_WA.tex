%=============================================================================%
%                      Compte rendu analyse non supervisé                     %
%=============================================================================%
\documentclass[12pt,a4paper]{article}
%=============================================================================%
%							PACKAGE											  %
%=============================================================================%
\usepackage[french]{babel} % langue
\usepackage[utf8]{inputenc} % encodage
\usepackage{a4} % format
\usepackage[T1]{fontenc}
\usepackage[cyr]{aeguill}
\usepackage{epsfig}
\usepackage{amsmath, amsthm}
\usepackage{amsfonts,amssymb}
\usepackage{float}

%=============== changement de la numérotation des section ect...=============%
% \renewcommand{\thesection}{\Alph{section}}  % ex: ici les section sont définis avec des lettre
% \Alph --> Lettre	;	\Roman --> chiffre romain
%============================= couleur texte =================================%
\usepackage{xcolor}
\definecolor{green2}{rgb}{0,0.6,0}
\definecolor{grey2}{rgb}{0.4,0.4,0.4}

%============================ lien hypertext =================================%
\usepackage{hyperref}	
%utilisation href{url}{texte contenant le lien hypertext}

%============================ marge document =================================%
\usepackage{geometry}
\geometry{hmargin = 1.5cm, vmargin = 2cm}

%============================ mode paysage ===================================%
\usepackage{lscape} % page en paysage

%=============================== En-têtes et pied de page ====================%
\usepackage{fancyhdr}
\pagestyle{fancy} % est un style de page.
\usepackage{lastpage}
\renewcommand\headrulewidth{1pt}
\fancyhead[R]{Gestion informatique des données de séquençage}
\fancyhead[L]{M1 BI-IPFB Université de Paris}
\renewcommand\footrulewidth{1pt}
\fancyfoot[l]{William Amory}
\fancyfoot[c]{\textbf{Page \thepage/\pageref{LastPage}}}
\fancyfoot[r]{\today}
%=============================================================================%
%					    	Titre - Auteur - Date	        	 		      %
%=============================================================================%
\title{Gestion informatique des données de séquençage \\ Genoscope - LBGB}
\author{William Amory \\ M1 BI-IPFB Université de Paris \\\\ sous la responsabilité de Frédérick Gavory}
\date{\today}
%=============================================================================%
%						        Début document   							  %
%=============================================================================%
\begin{document}
\maketitle % afficher titre
%\tableofcontents % afficher table des matières
%\newpage
%=============================================================================%
%================================== intro ====================================%
\section{Introduction} 


%=============================================================================%
%=============================== result ======================================%


%=============================================================================%
%============================== conclusion ===================================%


%=============================================================================%
%================================ Figures ====================================%


%=============================================================================%
%============================= reférences ====================================%
%\bibliographystyle{abbrv}
%\bibliography{nom_fichier}
%\nocite{*}
%=============================================================================%
%		      				Fin document		  							  %
%=============================================================================%
\end{document}

\begin{minipage}{0.45\textwidth}
	\begin{figure}[H]
		\centering
		\includegraphics[width=1\textwidth]{img/optimal_clusters_HCA.png}
		\caption{Détermination graphique du nombre de clusters optimal pour la classification à l'aide de HCA}
		\label{fig3}
	\end{figure}
\end{minipage} 
\hfill
\begin{minipage}{0.5\textwidth}
	\begin{figure}[H]
        \centering
        \includegraphics[width=1\textwidth]{img/heatmap.png}
        \caption{Heatmap de la classification des poches et des descripteurs par HCA}
        \label{fig4}
    \end{figure}
\end{minipage}

\begin{figure}[H]
    \centering
    \includegraphics[width=0.45\textwidth]{img/poucentage_variance_composante_pca.png}
    \includegraphics[width=0.45\textwidth]{img/pca_biplot_hca.png}
    \caption{Détermination graphique du nombre de composante nécessaire (à gauche) et visualisation des clusters sur les 2 premières composantes de l'ACP (à droite)}
    \label{fig5}
\end{figure}