\section{Perspectives}

\subsection{Workflow MGI}
Concernant le workflow de MGI, il nous faut dans un premier temps déterminer les outils et méthodes necessaires (utilisation de ceux du workflow d'Illumina ou de nouveaux). Une fois ceci déterminé il restera à écrire les deux pipelines, celui de génération de reads (ngs\_rg\_mgi) et celui de contrôle qualité (ngs\_rg\_mgi). L'objectif sur le long terme est d'arriver à un workflow totalment automatisé, comme celui d'Illumina.\\

\subsubsection{Le pipeline NGS\_RG\_MGI}
\textcolor{red}{ajout démidage, maintient pipeline. ??????????}

\subsubsection{Le pipeline NGS\_QC\_MGI}
\textcolor{red}{maintient pipeline. ?????????}

\subsection{Évaluation d'outils}
Il y aura aussi l'évaluation d'autres outils utiles pour les pipelines, comme des outils de \emph{trimming}, \emph{filtering}, d'assignation taxonomique, etc.