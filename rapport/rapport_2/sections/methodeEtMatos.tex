\section{Matériels et Méthodes}
\subsection{Le cluster de calcul et Slurm}
Le Genoscope possède 12 noeuds de calculs pour la \emph{production} sur le nouveau cluster \emph{inti}, ces derniers disposent de 16 cœurs et de 257 Go de RAM(mémoire vive). L'accès à l'utilisation des clusters est réalisé par le logiciel Slurm\footnote{Logiciel open source d'ordonnancement des tâches informatiques}.

\subsection{La base de données de référence NGL et la gestion des projets}
Le Genoscope dispose de sa propre base de données de référence NGL. Celle-ci est divisée en plusieurs parties. NGL\_BI\footnote{NGL Bioinformatic}, est la partie de la base de données utilisée par les équipes de bioinformatique. NGL\_SEQ\footnote{NGL Sequencing}, est la partie de la base de données utilisée dès la réception des échantillons et jusqu'au séquençage de ces derniers. Il y a également les parties NGL\_sub\footnote{NGL submission (base de données des job soumis aux clusters)}, NGL\_reagent\footnote{NGL reagent (base de données des réactifs)} et NGL\_projects\footnote{NGL projects (base de données des projets en cours et passé)}. La gestion et le suivi du développement informatique sont réalisés par le système de tickets Jira\footnote{Logiciel de gestion de projet, d'incidents et de suivi de bugs}.

\subsection{Le langage de programmation Perl}
L'écriture du workflow des pipelines pour les séquenceurs MGI est réalisée dans le langage de programmation Perl. L'utilisation de ce langage est rendu necessaire pour des raisons historique du laboratoire, puisque de nombreuses librairies et modules qui ont été utilisé dans l'écriture des pipelines sont écrits en Perl.\\

C'est pour toutes ces raisons qu'il m'a été nécessaire d'apprendre à coder en Perl. j'ai donc commencé par réaliser un programme permettant de faire des analyses statistiques élémentaires sur des fichiers fastq, tel que le taux de GC, la moyenne du score de la qualité, ainsi que plusieurs autres métriques. Le programme est capable de gérer les fichiers fastq issue de séquençage \emph{single end\footnote{Lecture dans un seul sens des reads par le séquenceur}} et \emph{paired end\footnote{Lecture dans les deux sens des reads par le séquenceur}}. Cela m'a permis de prendre en main les librairies Perl utilisées pour les différents pipelines déja en place. Ainsi que de m'habituer à l'environement de travail, l'utilisation du lancement de job sur les noeuds de calculs et l'utilisation des modules\footnote{Un module contient un ou plusieurs logiciels tiers ou dévellopé par les équipes du genoscope. Il est néccessaire de les charger dans notre environement de travail pour pouvoir utiliser ces logiciels.} pour les différents pipelines.

\subsection{Évaluation de bcl2fastq et bcl-convert}
Une première évaluation de deux logiciels de génération de fichiers fastq et de démultiplexage, développés et commercialisés par Illumina a également été efféctuée. Cette évaluation a été nécessaire pour déterminer les changements qu'il y aura à faire dans les pipelines en vue du remplacement de bcl2fastq (qui sera bientôt obsolète) par bcl-convert.\\

Dans un premier temps, il a été nécessaire de déterminer les conditions optimales de bcl2fastq (temps total, temps CPU, pourcentage d'utilisation CPU en fonction des ressources disponibles sur les noeuds du cluster (\emph{inti}) réservé à la \emph{prodution}, avec l'objectif de pouvoir appliquer les mêmes conditions à bcl-convert. Les conditions optimales sont déterminées en fonction des paramètres suivants de bcl2fatq (l'équivalent de bcl-convert est indiqué entre crochets): \\
\begin{itemize}
    \item[•] \texttt{r} [bcl-num-decompression-threads] : nombre de \emph{threads\footnote{Processus : instructions du langage machine d'un processeur.}} accordé pour la décompréssion et la lecture des \emph{Bases Calls\footnote{Fichier d'attribution des bases nucléiques en fonction des pics du chromatogramme lors du séquençage}}
    \item[•] \texttt{p} [bcl-num-conversion-threads] : conversion des \emph{Bases Calls} en fastq
    \item[•] \texttt{w} [bcl-num-compression-threads] : écriture et compréssion des fichiers fastq\\
\end{itemize}

Tous ces tests ont été  réalisés sur le même noeud de calcul, dans l'objectif de minimiser les biais. La comparaison est effectuée sur le temps total de génération des fastq et le démultiplexage, ainsi que le temps CPU et le pourcentage d'utilisation des CPU.

\subsection{Le pipeline NGS\_RG\_MGI}
\textcolor{red}{descriptions des fichiers et outils à disposition pour réaliser le pipeline}

\subsection{Le pipeline NGS\_QC\_MGI}
\textcolor{red}{descriptions des fichiers et outils à disposition pour réaliser le pipeline}