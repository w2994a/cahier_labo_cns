\section*{Glossaire}

\begin{description}
    \item[BGI : ] \emph{Beijing Genomics Institute}, est une entreprise Chinoise de biotechnologie fondé en 1999.
    \item[CEA :] Commissariat à l'Énergie Atomique et aux Énergies Alternatives
    \item[CNRGH :] Centre National de Recherche en Génomique Humaine 
    \item[CNS :] Centre National de Séquençage (Genoscope)
    \item[CPU :] \emph{Central Processing Unit} (Unité Central de Traitement)
    \item[DRF :] Direction de la Recherche Fondamental
    \item[ERGA] \emph{European Reference Genome Atlas}
    \item[IBFJ :] Institut de Biologie François Jacob
    \item[Illumina :] Entreprise Californienne de biotechnologie fondé en 1998, qui réalise : R\&D, production et vente d’instruments de séquençage d’ADN, de réactifs et de produits connexes
    \item[Jira :] Logiciel de gestion de projet, d'incidents et de suivi de buds développé par l'entreprise Atlassian
    \item[LBGB :] Laboratoire de Bioinformatique pour la Génomique et la Biodiversité
    \item[LIMS :] \emph{Laboratory Information Management System}
    \item[MGI : ] Filiale du groupe BGI fondé en 2016 dont les missions sont : R\&D, production et vente d’instruments de séquençage 
    d’ADN, de réactifs et de produits connexes
    \item[NCBI :] \emph{National Center for Biotechnologiy Information}, est un institut national des Etats Unis d'Amériques pour l'information biologique moléculaire. Il dévellope notament la base de données de génomes GenBank et la base de données des publications PubMed
    \item[NGL :] \emph{Next Generation LIMS}
    \item[NGS :] \emph{Next Generation Sequencing}
    \item[Oxford Nanopore :] Entreprise Anglaise de biotechnologie fondé en 2005, qui réalise : R\&D, production et vente d’instruments de séquençage d’ADN, de réactifs et de produits connexes
    \item[PacBio :] \emph{Pacific Biosciences} est une entreprise Californienne fondé en 2004, qui réalise : R\&D, production et vente d’instruments de séquençage d’ADN, de réactifs et de produits connexes
    \item[Path : ] Chemin d'accès à un fichier ou à un répertoire dans le système de fichier
    \item[PERL :] \emph{Pratical Extraction and Report Language}
    \item[RAM :] \emph{Random Access Memory} (Accès Mémoire Aléatoire)
    \item[Slurm :] \emph{Simple Linux Utility for Resource Management} qui est un logiciel open source d'ordonnancement des tâches informatiques
\end{description}