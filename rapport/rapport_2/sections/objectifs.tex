\section{Objectifs}
L'objectif principal de ma mission est la mise en place d'un workflow NGS pour les séquenceurs de MGI. Plus précissément il s'agira de créer un pipeline de génération de reads (ngs\_rg\_mgi\footnote{\emph{Next Generation Sequencing - reads generation - mgi}}) et un pour le contrôle qualité (ngs\_qc\_mgi\footnote{\emph{Next Generation Sequencing - quality control - mgi}}). Le workflow devra créer et mettre à jour l'état des runs et des \emph{lanes}\footnote{pistes présentes sur la \emph{flow cell}} dans ngl, réaliser le contrôle qualité des fichiers de séquence, au format fastq, obtenu après démultiplexage\footnote{Séparation des différents \emph{reads} d'une \emph{lane} en fonction de l'index d'échantillon} des runs. Il devra mettre à jour l'avancement du traitement d'un run dans NGL, en y insérant les statistiques obtenues lors du démultiplexage, les résultats des contrôles qualités, etc. Puisque l'objectif est d'obtenir un premier niveau de valorisation des fichier de séquences, permettant aux autres groupes (\og assemblage \fg{}, \og anotation \fg{}) de prendre en charge ces fichiers avant de les mettrent à disposition des laboratoires collaborateurs.\\

Je dois également, rechercher et réaliser des évaluations de nouveaux outils pour les différents pipelines des différentes technologies de séquençage. En vue d'un potentiel ajout d'outils ou de remplacement d'outils. Il sera donc necessaire de maintenir les pipelines des différentes technologies de séquençage en conséquence.