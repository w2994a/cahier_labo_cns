\section{Objectifs de ma mission}
L'objectif principal de ma mission est la mise en place d'un workflow NGS pour les séquenceurs MGI. Dans un premier temps il s'agira de créer un pipeline de génération de fichiers de séquences (NGS\_RG\_MGI\footnote{\emph{Next Generation Sequencing - reads generation - mgi}}) puis un pour le contrôle qualité de ces fichiers (NGS\_QC\_MGI\footnote{\emph{Next Generation Sequencing - quality control - mgi}}). Le workflow devra créer et mettre à jour l'état des runs, des pistes ou (\emph{lane}) et de readset\footnote{Lot de séquences} dans NGL, réaliser le contrôle qualité des fichiers de séquences, au format FASTQ, obtenus en fin de séquençage. Il devra mettre à jour l'avancement du traitement d'un run dans NGL, en y insérant les métriques et statistiques obtenues lors du démultipléxage\footnote{Séparation des séquences en plusieurs fichiers en fonction de leurs index (séquence d'une dizaines de nucléotides en amont du primer de la séquence)}, les résultats des contrôles qualités, etc. Puisque l'objectif est d'obtenir un premier niveau de valorisation des fichiers de séquences, permettant aux autres groupes (\og assemblage \fg{}, \og annotation \fg{}) de prendre en charge ces fichiers avant de les mettre à disposition des laboratoires collaborateurs.\\

Je dois également, rechercher et réaliser des évaluations de nouveaux outils pour les différents pipelines des différentes technologies de séquençage, en vue d'un potentiel ajout ou de remplacement d'outils. Il sera donc nécessaire de maintenir les pipelines des différentes technologies de séquençage en conséquence. Par exemple l'évaluation de logiciels de trimming (Cutadapt, Trimmomatic) en vue d'un remplacement du l'outil fastx\_clean de l'extension fastxend de la suite FASTX Toolkit\footnote{Collection de commandes pour le traitement et l'évaluation de lot de séquences au format FASTA ou FASTQ} qui est un outil mono-coeur pour un outil multi-coeurs. Ou bien trouver et évaluer un logiciel d'assignation taxonomique plus performant que le logiciel Centrifuge utilisé actuellement.