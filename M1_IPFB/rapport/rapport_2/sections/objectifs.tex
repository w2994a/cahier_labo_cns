\section{Objectifs de ma misssion}
L'objectif principal de ma mission est la mise en place d'un workflow NGS pour les séquenceurs de MGI. Plus précissément il s'agira de créer un pipeline de génération de fichiers de séquences (ngs\_rg\_mgi\footnote{\emph{Next Generation Sequencing - reads generation - mgi}}) et un pour le contrôle qualité de ces fichiers (ngs\_qc\_mgi\footnote{\emph{Next Generation Sequencing - quality control - mgi}}). Le workflow devra créer et mettre à jour l'état des runs, des \emph{lanes}\footnote{pistes présentes sur la \emph{flowcell}} et de \emph{readset}\footnote{Lot de séquences} dans NGL, réaliser le contrôle qualité des fichiers de séquences, au format fastq, obtenus après démultiplexage\footnote{Séparation des différents \emph{reads} d'une \emph{lane} en fonction de l'index d'échantillon} des runs. Il devra mettre à jour l'avancement du traitement d'un run dans NGL, en y insérant les statistiques obtenues lors du démultiplexage, les résultats des contrôles qualités, etc. Puisque l'objectif est d'obtenir un premier niveau de valorisation des fichiers de séquences, permettant aux autres groupes (\og assemblage \fg{}, \og anotation \fg{}) de prendre en charge ces fichiers avant de les mettrent à disposition des laboratoires collaborateurs.\\

Je dois également, rechercher et réaliser des évaluations de nouveaux outils pour les différents pipelines des différentes technologies de séquençage. En vue d'un potentiel ajout ou de remplacement d'outils. Il sera donc necessaire de maintenir les pipelines des différentes technologies de séquençage en conséquence. Par exemple l'évaluation de logiciels de trimming (Cutadapt, Trimmomatic) en vue d'un remplacement du logiciel fastx_clean de la suite FASTX Toolkit\footnote{Collection de commandes pour le traitement et l'évaluation de lot de séquences au format FASTA ou FASTQ} qui est un outil mono-coeur pour un outil multi-coeurs. Ou bien trouver et évaluer un logiciel d'assignation taxonomique plus performant que le logiciel Centrifuge utilisé actuellement.