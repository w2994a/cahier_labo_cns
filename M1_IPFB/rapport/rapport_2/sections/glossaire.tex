\section*{Glossaire}

\begin{description}
    \item[BGI : ] \emph{Beijing Genomics Institute}, est une entreprise Chinoise de biotechnologie fondé en 1999.
    \item[CEA :] Commissariat à l'Énergie Atomique et aux Énergies Alternatives
    \item[CNRGH :] Centre National de Recherche en Génomique Humaine 
    \item[CNS :] Centre National de Séquençage (Genoscope)
    \item[CPU :] \emph{Central Processing Unit} (Unité Central de Traitement)
    \item[DNB :] \emph{DNA-nanoballs} (Nano \og billes\fg{} d'ADN générés lors de l'amplification ADN pour les séquenceurs de la technologie MGI)
    \item[DRF :] Direction de la Recherche Fondamentale
    \item[ERGA] \emph{European Reference Genome Atlas}
    \item[IBFJ :] Institut de Biologie François Jacob
    \item[Illumina :] Entreprise Californienne de biotechnologie fondée en 1998, qui réalise : R\&D, production et vente d’instruments de séquençage d’ADN à haut débit et très haut débit, ainsi que des logicels et services d'anlyses bio-informatique des données de séquençage.
    \item[Jira :] Logiciel de gestion de projet, de suivi d'incidents et de bugs développé par l'entreprise Atlassian
    \item[LBGB :] Laboratoire de Bioinformatique pour la Génomique et la Biodiversité
    \item[Lims :] \emph{Laboratory Information Management System}
    \item[MGI : ] Filiale du groupe BGI fondée en 2016 dont les missions sont : R\&D, production et vente d’instruments de séquençage 
    d’ADN, de réactifs et de produits connexes
    \item[NCBI :] \emph{National Center for Biotechnologiy Information}, est un institut national des Etats Unis d'Amériques pour l'information biologique moléculaire. Il dévellope notament la base de données de génomes GenBank et la base de données des publications PubMed
    \item[NGL :] \emph{Next Generation LIMS} (bases de données du Genoscope et du CNRGH)
    \item[NGL\_BI :] \emph{NGL Bioinformatic} (base de données des analyses et traitements bio-informatique)
    \item[NGL\_PROJECT :] \emph{NGL projects} (base de données des projets en cours et passé)
    \item[NGL\_REAGENT :] \emph{NGL reagent} (base de données des réactifs)
    \item[NGL\_SEQ :] \emph{NGL Sequencing} (base de données de suivi des échantillons)
    \item[NGL\_SUB :] \emph{NGL submission} (base de données des soumissions de projet ou d'articles (exemple : la soumission d'un projet au NCBI))
    \item[NGS :] \emph{Next Generation Sequencing}
    \item[NGS\_BA :] \emph{Next Generation Sequencing - biological analysis}
    \item[NGS\_QC] \emph{Next Generation Sequencing - quality control}
    \item[NGS\_RG : ] \emph{Next Generation Sequencing - reads generation}
    \item[Oxford Nanopore :] Entreprise Anglaise de biotechnologie fondée en 2005, qui dévellope et produit des systèmes de séquençage, basé sur les propriété diélectrique de ces dernières.
    \item[PacBio :] \emph{Pacific Biosciences of California} est une entreprise Californienne fondée en 2004, qui dévellope et produit des systèmes de séquençage en temps réel à molécule unique (SMRT) d'ADN
    \item[Path : ] Chemin d'accès à un fichier ou à un répertoire dans le système de fichier
    \item[Perl :] \emph{Pratical Extraction and Report Language}
    \item[Ram :] \emph{Random Access Memory} (Accès Mémoire Aléatoire, aussi appelé mémoire vive)
    \item[Slurm :] \emph{Simple Linux Utility for Resource Management} qui est un logiciel open source d'ordonnancement des tâches informatiques
\end{description}