%=============================================================================%
%                                 Preamble                                    %
%=============================================================================%
\documentclass[12pt,a4paper]{article}
%=============================================================================%
%							PACKAGE											  %
%=============================================================================%
\usepackage[french]{babel} % langue
\usepackage[utf8]{inputenc} % encodage
\usepackage{a4} % format
\usepackage{setspace} % interligne
\onehalfspacing % définit un interligne de 1.5
\usepackage[T1]{fontenc}
\usepackage[cyr]{aeguill}
\usepackage{epsfig}
\usepackage{amsmath, amsthm}
\usepackage{amsfonts,amssymb}
\usepackage{float}

\usepackage{multirow} % tableau avancé

\usepackage{endnotes} % note en fin de document
\let\footnote = \endnote % redéfinit la commande \footnote en \endnote 
\renewcommand{\notesname}{Notes} % Rennomer le titre des endnotes.

% \renewcommand{\thefigure}{\ifnum\value{section}>0 
% 	\thesection.\fi\arabic{figure}} % redéfinit la numérotation des figures en fonction des section
% \renewcommand{\thetable}{\ifnum\value{section}>0 
% 	\thesection.\fi\arabic{figure}} % redéfinit la numérotation des tables en fonction des section
%=============== changement de la numérotation des section ect...=============%
% \renewcommand{\thesection}{\Alph{section}}  % ex: ici les section sont définis avec des lettre
% \Alph --> Lettre	;	\Roman --> chiffre romain
%============================= couleur texte =================================%
\usepackage{xcolor}
\definecolor{green2}{rgb}{0,0.6,0}
\definecolor{grey2}{rgb}{0.4,0.4,0.4}

%============================ lien hypertext =================================%
\usepackage{hyperref}	
%utilisation href{url}{texte contenant le lien hypertext}

%============================ marge document =================================%
\usepackage{geometry}
\geometry{hmargin = 2cm, vmargin = 2cm}

%============================ mode paysage ===================================%
\usepackage{lscape} % page en paysage

%=============================== En-têtes et pied de page ====================%
\usepackage{fancyhdr}
\pagestyle{fancy} % est un style de page.
\usepackage{lastpage}
\renewcommand\headrulewidth{1pt}
\fancyhead[R]{Gestion informatique des données de séquençage}
\fancyhead[L]{M1 BI-IPFB Université de Paris}
\renewcommand\footrulewidth{1pt}
\fancyfoot[l]{William Amory}
\fancyfoot[c]{\textbf{Page \thepage/\pageref{LastPage}}}
\fancyfoot[r]{\today}